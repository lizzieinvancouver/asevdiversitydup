\documentclass[11pt]{article}
\usepackage[top=1.00in, bottom=1.0in, left=1.1in, right=1.1in]{geometry}
\renewcommand{\baselinestretch}{1.1}
\usepackage{graphicx}
\usepackage{natbib}
\usepackage{amsmath}
\usepackage{parskip}

\def\labelitemi{--}
\parindent=0pt


\begin{document}
\bibliographystyle{/Users/Lizzie/Documents/EndnoteRelated/Bibtex/styles/besjournals}
\renewcommand{\refname}{\CHead{}}

\title{Dangers of limited genetic diversity to the world's terroir \\ OR \\ Dangers of limited winegrape diversity in the Anthropocene}
\author{Lizzie, hopefully with M. A. Walker, I. Garc\'ia de Cort\'azar-Atauri \& T. Lacombe}
\date{\today}
\maketitle
% Institut National de la Recherche Agronomique (INRA), US 1116 AGROCLIM, F-84914 Avignon, France \\


% To do ...
% Do a QUICK review of REFS and XX and fill in as easily possible; leave the rest to ask co-authors about
% Send to co-authors. 
% Find cites on downy and powdery mildew
% ENTAV clone refs?
% Work on fixing the end bit 

% Refs to add:
% G x E x M somewhere: cooper2021
% Tomato breeding: tao2014


\begin{abstract}
Climate change alongside shifts in management regimes means a new landscape of potential yield losses to due disease, pests and weather for most winegrape growers. This new world has elevated the importance of exploiting the genetic diversity of winegrapes and wild \emph{Vitis} relatives. It has also highlighted the challenge of terroir. While conceptually terroir captures how much the environment can be expressed through the wine produced, scientifically it highlights how much winegrape traits shift dynamically with their environment---effectively how much environmental interactions underpin winegrowing and should underpin winegrape research. Here we review the genetic diversity of winegrapes, and its critical connection to the phenotypic diversity necessary to adapt to shifting pressures in the decades and centuries to come. We then outline that a globally predictive framework for winegrowing has been slow because research does not always include the preeminence of environmental interactions in winegrapes. Addressing this challenge could revolutionize winegrape growing, but will require new approaches to collecting and synthesizing data. We argue the opportunity and need for this transition has never been greater, as climate change means each region is shifting to a new region over time, making the need to build a predictive framework that works across regions increasingly urgent. % While 150 years ago European winegrapes were saved by North American \emph{Vitis} diversity---through its pest resistance---today's diversity to save winegrapes from another global threat is much smaller. 
\end{abstract}


\newpage
\section{Main text}
\emph{Introduction}

As the pace of anthropogenic climate change becomes ever clearer, impacts on winegrowing regions have increased. Growers are dealing with shifts in frost, storms, drought, heat and associated fires, alongside shifts in pests and pathogens. At the same time shifts towards more sustainable management, including organic and resilient methods, may increase losses to pest an disease \citep[but reduce other environmental costs,][]{doering2019,vanderwerf2020}. Combined these changes mean a new landscape of losses to due disease, pests and weather for most growers globally.   % the prevalence and distribution of pests and pathogens

In such a brave new world, the importance of exploiting the genetic diversity of winegrapes---and wild \emph{Vitis} relatives---has never been higher. In some regions interest in previously uncommon varieties, which may be more drought or heat tolerant, is on the rise, while in other regions growers are increasingly looking towards hybrid grape varieties for disease resistance. These shifts are occurring alongside growing insights into the history of the domestication of winegrapes \citep[e.g.,][]{ramosmad2019,dong2023}. % , rising up out of the Fertile Crescent alongside major cereal and other crops pivotal to human civilization

Yet, while these shifts hold promise for a future winegrape crop that is more diversified and richer---both genetically and phenotypically---the crop today has low global diversity \citep{Wolkovich2017,andersonnelgen2021}. Further, we are likely losing diversity continually as relic populations of ancient varieties, wild relatives of the winegrape, and broader grape diversity are lost to habitat change and shifts in agriculture. Such losses could limit how well winegrowing regions can adapt to higher levels of climate change and shifts in pests and diseases. 

Here we review the genetic diversity of winegrapes, and its connection to the phenotypic diversity necessary to adapt to shifting pressures in the decades and centuries to come. Our focus spans from \emph{Vitis vinifera} subsp. \emph{vinifera} clones of the world's most planted varieties (e.g., Chardonnay and Pinot noir) to the wild \emph{Vitis} species that underlie most of the world's winegrowing today (Fig. \ref{fig:vitislevels}). We outline the potential for losses in the future at a time when diversity is most needed. Exploiting that diversity, however, will require new efforts to understand environmental variation across levels alongside how constraints on critical traits, such as drought tolerance and production, vary. % We outline a path forward that will be rad. 
% While the crop of winegrapes has already experienced major losses of diversity, we outline the potential for losses in the future at a time when diversity is most needed.

\subsection{Diversity across time \& space}

\emph{From species to clones and back again: A history of winegrape diversity}
% To do:
% Add in variety diversity (somewhere before/around 'until the 1500s) ??
% Watch out -- I fear I am getting a little long in the story here... I do NOT want a long paper. 

Europe historically has had one wild grape species (\emph{Vitis sylvestris}), from which winegrapes (\emph{Vitis vinifera} subsp. \emph{vinifera}) evolved (Fig.  \ref{fig:history}). Current evidence suggest two domestication events, one in the area of the Fertile Crescent and the other in the Caucasus, approximately 11,000 years ago, with a far greater development in the Fertile Crescent \citep{arroyo2006,mcgovern2017,riaz2018,dong2023}. This genetic pool mixed until the 1500s as an effectively closed system with no new genetic material \citep[e.g.,][]{dong2023}. 

This closed system ended in the 1500s with the European discovery of the Americas and its comparatively species-rich \emph{Vitis} diversity. North America's grapes (e.g., fox, \emph{V. labrusca}, and river, \emph{V. riparia}, grapes) were rapidly imported to Europe, as returning voyagers exclaimed over their fruit production \citep{campbell2005vitner}  % http://www.hort.cornell.edu/reisch/grapegenetics/bulletin/wine/winetext3.html

But the increasing import of North American \emph{Vitis} also brought the near death-toll of winegrapes. After thousands of years of domestication and winegrowing, downy and powdery mildew appeared on winegrapes for first time. As growers struggled with this new disease pressure, they imported greater amounts of North American \emph{Vitis}, hoping to find a way to exploit the disease resistance from the very plants that carried in these new diseases. Importers unknowingly brought in an even greater threat: \emph{Phylloxera} \citep{galet1982,campbell2005vitner}. % could also look up The Great Wine Blight in Zurich (Ordish)

\emph{Phylloxera} spread across Europe. After various remedies, from tarring plants to flooding vineyards failed, North American  \emph{Vitis} provided a solution through its rootstock \citep{campbell2005vitner,galet2015}. Today most---but not all---of the world's winegrapes are grafted \emph{Vitis vinifera} subsp. \emph{vinifera} varieties on top (scion) of hybrid rootstocks, which include a large percentage of North American  \emph{Vitis} and, with it,  \emph{Phylloxera} resistance. 

As the world's winegrowing regions went through their own \emph{Phylloxera} epidemics and rebirths on grafted vines, Europe---as the first to succumb---was also the first to rebuild. By the early 1900s much of the variety structure of the regions we know today was established and New World regions increasingly took cues from Europe in which varieties to grow and where. New World regions also pushed to new climatic ranges for winegrowing, beyond the relatively narrow climate space of Europe. Cool climate regions in particular focused on hybrids developed in early to mid 20th century for their greater cold hardiness, while some warmer and wetter regions looked to hybrids for disease resistance (need REFS). % (which decades? Look up Marchel Foch)

At the same time that growers in the northern range edge were increasing the genetic diversity of winegrapes, growers in more traditional regions of Europe increasingly focused on a much narrower level of diversity---clones (unique genotypes produced through rare somatic mutations). Clones became one way for growers to refine the match of what they grew to their local environment. As European growers found certain clones appeared to perform best for their areas, New World growers often tried the same ones---with varying success. The reasons for this include the complex interaction between clones and varying environments (soil, climate etc.), but may also the complexity of what exactly a clone represented. 

With the rise of new genetic techniques, researchers could test---for the first time---whether clones were unique genetically. When they did, they found a wide number of well-loved clones were actually identical less loved ones, but varied in disease load. The clonal revolution, including new ENTAV (Etablissement National Technique pour l’Amelioration de la Viticulture) clones, of the 1970s lead to widespread shifts in what clones were planted and was effectively the last great shift in the genetics of the world's winegrowing regions until recent decades.  % https://www.vignevin.com/linstitut/histoire-ifv/
% Clones were traditionally selected by growers for their unique phenotypes, and today are defined as somatic mutations, but until the later 20th century, the technology to test for such difference  ....

[Need help with REFS if we want to keep above paragraphs.]\\

\emph{From species to clones and back again: A history of diversity lost}  %Or, A history of loss that continues today

The history of winegrape domestication is also a history of diversity lost. Domestication brought selective sweeps and a loss of genetic diversity as alleles fixed for traits important to human cultivation \citep{dong2023}. Today, few populations of \emph{vinifera's} wild ancestor, \emph{sylvestris}, remain. The \emph{Phylloxera} epidemic also led to losses of \emph{Vitis vinifera} subsp. \emph{vinifera}. While researchers and growers raced to save genetic material (for example, Montpellier scientists established the Domaine de Vassal in the \emph{Phylloxera}-free sands of southern France), varieties that were less common or endemic to hard-hit areas were lost. While some may remain today, especially in regions still untouched by \emph{Phylloxera} (Andy, any REFS for Chile or such?), missing links in the  \emph{Vitis vinifera} subsp. \emph{vinifera} family tree \citep[e.g.,][]{merlotparents} must come in part from this period. 
% The post \emph{Phylloxera} era, while omnipresent in winegrowing today, is only a blip in the history of winegrapes.

Beyond Europe, multiple threats have reduced the diversity of North American and Asian species. Habitat loss and fragmentation have reduced populations of most species, especially \emph{Vitis rupestris} and some \emph{Vitis cinerea} varieties \citep{heinitz2019}. Further, because of the widespread planting of  \emph{Vitis vinifera} subsp. \emph{vinifera} and because \emph{Vitis} species interbreed easily, many North American species have lost diversity through hybridization with \emph{vinifera} \citep{heinitz2019}. Such trends continue today. This loss may be especially problematic as these species underpin rootstock, providing critical resistance traits to disease, drought and pests, while also being poorly documented and conserved compared to \emph{Vitis vinifera} subsp. \emph{vinifera}. 

\emph{Is Chardonnay our future diversity?}

Loss of \emph{Vitis vinifera} subsp. \emph{vinifera} globally continues. Rare varieties still hidden in small vineyards or less developed regions, beyond the grasp of research collections, are lost to replanting or other shifts. There is currently relatively little incentive to save such vineyards, as globally the wine market focuses increasingly on a few select varieties \citep{Wolkovich2017,andersonnelgen2021}. International varieties---so named for how widespread growers plant them---represent 70-80\% of major New World growing regions \citep{Wolkovich2017} and have grown from representing 14\% to 38\% of globally planted hectares since 1990 \citep{andersonnelgen2021}. The result is that most \emph{vinifera} diversity remains locked in the Old World where it originated, as few growers risk trying to grow---and sell consumers on---new varieties.  % Some of the youngest regions, for example China, currently plant upwards of XX hectares with one variety. 

Further, climate change has brought the possibility of growing \emph{vinifera} grapes to regions that previously didn't, such as parts of Canada and the United Kingdom, where hybrid grapes have declined and plantings of \emph{vinifera}, especially international varieties, have increased \citep{Wolkovich2017}. While the flavor quality of hybrid varieties has increased, there is no clear trend that the planting of hybrids has increased (REFS). 

Declining winegrape diversity presents a major threat to the sustainability of the crop. While 150 years ago European winegrapes were saved by North American \emph{Vitis} diversity---through its pest resistance---today's diversity to save winegrapes from another global threat is much smaller. This decreased diversity comes at exactly the time when multiple threats loom, including drought and heat stress due to climate change, the rise of new pests or the spread of current disease (e.g., grapevine trunk disease), alongside other currently unknown threats \citep{Gramaje2018}. 

\subsection{Why diversity matters: traits, environments \& interactions}

\emph{Trait diversity from the Fertile Crescent to the Anthropocene}

Genetic diversity in all crops matters, as it underlies the traits that make crops valuable to humans. Both winegrape growers and breeders have continually selected on a suite of major traits that affect both how easy it is to grow grapes, including crop consistency, disease and pest tolerance, and the type of juice such grapes produce, including berry size and color, and crop load \citep{dong2023}. Alongside this, selection for phenology, cold and heat tolerance have produced grapes that are more or less feasible to grow depending on a region's climate.  

With climate change the feasibility of different winegrapes in different regions has shifted as phenology has advanced \citep{webb2012,Malheiro2013,vanlee2016oeno}, and reports of heat and drought stress have increased in many growing regions \citep{blancoward2019,diCarlo2019}. The Anthropocene era of winegrowing has thus raised the importance of climatic traits while a growing shift to more environmentally sustainable practices in some regions has increased the need for pest and disease resistance \citep{doering2019,merot2020,vanderwerf2020}. Because variation in traits is not constant across different levels of diversity, nor across different parts of the \emph{Vitis} phylogenetic tree, how best to exploit genetic diversity for sustainable winegrowing during the Anthropocene is not straightforward. 
% Do I need to add more here on where variation in different traits appear? Or just add sentence saying "variation in these things occurs at different levels ... see figure" ?
% Link this text to Figure: organize the levels of diversity and highlight where we have a lot of important variation -- for example, at variety level we do have phenological diversity and maybe some heat and drought tolerance, but we have little pest/disease tolerance

Various research teams have emphasized the power of clones, \emph{vinifera} varieties or hybrids to bring long-term resilience to winegrowing regions \citep[e.g.,][]{Myles2013,duchene2016,Wolkovich2017}. The best approach depends in part on the most important traits to a growing region (or grower) and what consumers will accept. It also, however, depends on the actual region (and vineyard) itself because of the tremendous levels of environmental variation present in the crop of winegrapes. 

\emph{Terroir \& the tyranny of environmental variation}

Terroir---the concept of how the full suite of environmental traits, combined with the varieties planted, define the wines of a region---is both a blessing and a curse in many ways to winegrowing. The concept of terroir captures how much the environment can be expressed through the wine produced. But this close connection between wine and its environment also emphasizes how much the environment determines the winegrapes produced, effectively highlighting that winegrape traits are not constant across space or time, but shift dynamically with their environments. 

This winegrape by environment interaction is part of the concept of terroir, and also a common concept in biology---referred to as a genotype by environment (G x E) interaction. G x E interactions underlie why the same variety has a much shorter growing season length in a hot climate compared to a cool climate, or why berry size varies in the same vineyard in a wet versus dry year. These two examples are some of the more well understood G x E examples. Winegrapes as a crop likely have high levels of such interactions and their complexity is multiplied by the chimeric form of most winegrape vines today: the rootstock and scion. 

Researchers and growers have studied rootstock by scion interactions (R x S) since rootstock's widespread introduction after the Phylloxera epidemic \citep[e.g.,][]{pouget1982,lupe1988,gautier2020,marguerit2012,tandonnet2010}. Some R x S can produce wildly unexpected scion responses (Examples? REFS?), but often the outcome is much more dependent on environmental conditions. Certain rootstocks work best in drier or more calcium-rich soils---a rootstock x environment (R x E) interaction. But these interactions themselves are not always stable. Thus, any one region will generally have suggested pairings depending on the scion variety, the soil and other environmental factors---and these often vary from region to region, showing how common R x S x E interactions are. 

Despite the central importance of rootstock to managing environmental variation a framework to understand and predict R x S x E has never emerged. This is likely due to the extreme complexity of interactions, and the sheer size of research needed to estimate them. Estimating a two way interaction (R x S, or G x E, for example) statistically takes generally 16X more samples than estimating one effect alone (assuming the interaction is half as large as the single effect), and estimating a three-way interaction takes exponentially more samples, depending on how large the estimated effect is \citep{regotherstories}. Not surprisingly then, these interactions have never been estimated for winegrapes, to our knowledge. 
% https://statmodeling.stat.columbia.edu/2018/03/15/need-16-times-sample-size-estimate-interaction-estimate-main-effect/

The data to estimate R x S x E effects has been collected, however. R x S studies are increasing, and have now been conducted in research experiments in at least XX regions (REFS, Fig. \ref{fig:rootstockxscionexpts}). Given overlap in rootstock and scions (CHECK), such experiments have collected the data to estimate R x S x E (Fig. \ref{fig:rootstockxscionexpts}). Estimating these effects, however, would require robust sharing of data across research teams, which is not always the norm, though that is changing. 

A more robust way to estimate interactions is through distributed experiments, which have become widespread in ecology over the last 10 years and have long been used to standardize some crops, including winegrapes (e.g., GREVES in France). Distributed experiments are specifically designed to estimate how consistent effects (in ecology these often specifically interaction effects) are across environments. The design of the experiments is standardized across locations for a more reliable comparison, but each site usually only needs a minimal amount of space: for example, for R x S interactions a distributed experiment may suggest 4 rootstocks and 4 scions, resulting in 16 treatments with a small number of replicates per treatment. They then use the power of replication across space to overcome the sample size needs for estimating interactions. Such experiments in ecology have been widely successful in providing global answers to how major questions, and are now increasing in both the global reach of sites involved, and the questions they address \citep{nutnet2014}. 
% NutNet
% Inaki says:  there are usually experiments to test this situation in different crops - they evaluate the stability and homogenity of the crop under different environments (See page 3 for overview of what they do: https://www.geves.fr/wp-content/uploads/23-EN-Grandes-cultures-1.pdf)

% \item Plasticity in traits (never really sure what that meant with phenology, but maybe the idea that the forcing cue could vary based on environmental conditions. 

\subsection{A less tyrannical future for resilient winegrowing}

To date most winegrowing regions have dealt with the interactive complexity of growing grapes through ad hoc vineyard trials---of different rootstocks, or varieties, for example---followed by word of mouth on what works. But the level of environmental variation in winegrapes as a crop and the shifting climatic and disease landscape of the Anthropocene may make this approach untenable. 

Building resilient winegrowing regions would benefit dramatically from a concerted global effort towards a framework that identifies trait variation across the different levels of winegrape---and \emph{Vitis}---diversity and predicts them based on the environment. Such a framework would allow growers to more predictably select the best rootstock and scions for their current---and future---terroir. It would also ideally allow researchers to predict variation without measuring every genotype in every environment, allowing them to predict which unmeasured varieties, for example, may be most drought-tolerant or would yield a suite traits especially coveted by growers (heat and drought tolerance with rapid phenology, for example). 
% Above -- do we do enough to explain space for time variation with climate change (if we understand spatial environmental variation enough, can predict forward)? 

% A globally predictive framework for winegrowing is currently, however, a long way off. 
Working towards a globally predictive framework for winegrowing will require researchers to tackle a suite of challenging, but critical, questions, including:

\begin{enumerate}
\item \emph{What is the scale of variation in major traits at each level of diversity (clones/varieties/species)?} 

This is a fundamental biological question where the answer is, usually, that variation for most traits is smallest at the clonal level, higher at the variety level and highest across different species. For winegrapes recent work has examined the potential of clones to provide useful variation but it appears, as expected, limited (REFS). Beyond clones, the usual biological answer of greater diversity from varieties to species may not apply across all traits, as domestication has potentially increased the diversity of some traits, especially those related to fruit. 

Even in cases where trait diversity is higher across species than within \emph{vinifera} a greater appreciation of the variation at each level would help. Given consumer hesitancy  towards any small percent of  \emph{vinifera} genotypes in their wine, a better understanding of the sweep of variation growers miss out on when focusing on  \emph{vinifera} varieties could help sway some folks (or something). 

Researchers, especially breeders, have long tackled aspects of this question---and data to address it for some traits is certainly available. Phenology, for example, varies by up to 5 weeks for maturity across \emph{vinifera} varieties \citep{bours1995}, which may be on par with variation across some species, though we generally lack comparable phenological data for many \emph{Vitis} species. In contrast, for certain traits such as cold hardiness, \emph{vinifera} has almost no major variation when compared to variation across species. Certainly the climatic diversity that \emph{Vitis} species span will generally have driven greater diversity across species in many climatic traits, such as drought and heat tolerance. Breeders recognize this and have a suite of traits they know they will find in greater diversity in wild \emph{Vitis} species than within  \emph{vinifera}. The challenge for breeders then often becomes understanding trait correlations, as selecting for one coveted trait---such as drought tolerance---may covary with selecting for less ideal traits such as slower growth. 

\item \emph{Trait correlations: What suites of traits are correlated?} 

Trait correlations are a well known outcome of evolution and a major hurdle for breeding. But beyond breeding, a fuller understanding of which traits are correlated is critical to help growers select what to plant for future climate and management regimes. Despite the central importance of this question, it is not well studied for \emph{Vitis}. Basic questions such as how correlated the length of difference phenophases are (e.g., do varieties that take a long time to advance from budburst flowering also progress slowly to veraison?) are not clearly answered. The reason for this, however, may be related to the next (and perhaps most critical) question.

\item \emph{What is the scale of variation due to interactions?} %  And .. are international varieties unique in strong variety responses and weak V x C so get similar responses across terroir? 

We argue that many of the previous questions are not well answered because of the preeminence of environmental interactions in winegrapes (as discussed above). Because so many winegrape traits vary from environment to environment, partitioning variation to different scales of diversity and estimating trait correlations becomes especially challenging. Findings in one site may not translate directly to other locations, making the effort required to address these questions seem extremely high. 

Estimating the relative effect of interactions is not an impossible task however. As we outline below, the basic setup of vineyards creates much of the experimental setup needed to tackle this question, and some of the data are already available. Using any current or future data, however, requires careful analytical approaches, including how best to integrate genetic data. 

Increasingly available genetic data could revolutionize models of trait variation and G x E interactions in viticulture. Because genetic data can provide a quantitative estimate of how related different genotypes are it allows models that can better partition out the effect of genotype from environment. The heterozygosity of winegrapes challenges some of the standard methods for this, but, increasingly, new methods and data allow building reticulate phylogenies (where multiple evolutionary pathways are possible) and mapping traits onto them. New models have also narrowed in on the history of \emph{vinifera} speciation and domestication, which could help breeders by understanding better how traits have been gained in the past. To be most useful though, we need more models of \emph{Vitis} speciation beyond \emph{vinifera}. % (we need a lot more info on native \emph{Vitis} to breed in useful traits)

 % The challenge then is to recognize the importance of these questions to sustainable winegrowing and thus overcome the current hurdles to answering them. 

\end{enumerate}

\subsection{A global distributed experiment already underway}

Rapid progress towards answering these questions is possible, as \emph{Vitis} research collections, research trials and vineyards provide much of the experimental setup needed. \emph{Vitis} research collections aimed at protecting and studying the grape germplasm exist in many winegrowing countries, variously financed by federal and local governments (e.g., Domaine de Vassal in France, IMIDRA in Spain, USDA germplasms in the USA). These collections are often setup from an experimental standpoint---with standardized planting and management methods---and thus within each is often the opportunity to estimate trait diversity across multiple levels. Funding towards such research, however, is too rarely supported at the level required to both maintain collections and gather data within them. It may thus require growers to help prioritize such research. 

Research trials have tested for rootstock x scion interactions for at least a century \citep{hussman1930}, providing the raw data to begin to understand how consistent such interactions are across locations. Addressing this, however, requires widespread sharing of data and an emphasis on the value of synthetic work---where the effort to gather and robustly analyse data is recognized. Funding regimes that focus on new data in the belief that new data are the major way to gain new knowledge, miss the power of data already collected to build to mechanistic models that can be predictive beyond one site, and outside historical climate regimes. 

This canalized focus on new data leaves pressing questions unanswered, but is not uncommon. In the sciences this problem is often tackled by explicit funding of synthesis work \citep[e.g., many government and non-profit `think tanks' are designed explicitly to gather and learn from existing data,]{baron2017}. \emph{Vitis} research in the Anthropocene appears poised to gain immensely from placing a higher value on synthetic research (e.g., meta-analyses) that analyzes data produced across environments. % Synthesis is also critical to recognizing major gaps and prioritizing future data collection.

With climate change each region is effectively shifting to a new region over time, making the need to build a predictive framework that works across regions increasingly clear. Before the Anthropocene, a stable climate meant local research trials could provide long-term information for growers. Today, growers should be increasingly interested in what trials in regions warmer, drier or more variable regions compared to their own are finding. This greatly raises the value of collecting and synthesizing data across regions.  

Commercial vineyards represent the greatest potential source of data needed to adapt viticulture to climate change. Each vineyard is a large-scale trial of rootstock and scion varieties, with varying levels of replication across vineyards within and across regions. While each vineyard may also have different planting and management approaches, newer models (e.g., Bayesian hierarchical) are specifically designed to partition out such vineyard-level variation from the effects of rootstock, scion and their environmental interactions (Fig. \ref{fig:divblocks}). If growers can also provide additional information on soil or related environmental characteristics it would provide major data to tackle the challenge of environmental interactions in viticulture. This could potentially build to a more mechanistic model of terroir. 
% Add time step return information to growers? Make open science perspective more explicit.

\section{References}
\bibliography{/Users/Lizzie/Documents/git/bibtex/LizzieMainMinimal}

% Progress towards addressing these questions and building a predictive framework requires a shift for both researchers and growers in how they collect and share data. 

\clearpage
\section{Figures (draft!)}

\iffalse
\begin{enumerate}
\item Levels of diversity through time, highlight what was lost when
\item Levels of diversity -- what traits vary at which levels? And where are we missing information on trait variation?
\item Phenology response surface design? 
\item R x S experiments on a map?
% \item Consider doing the GDD across sites and a few varieties from Sophia's work? (maybe ask about France/a couple places in Napa/Sonoma and Arterra data?) ... definitely should think on what this figure would look like before investing in asking how we can use the data. ... Could be too much work or a good way to show environmental variation. 
% \item Phenology of current and new international varieties in Europe or such (along with number of varieties per region)? Could finally publish that shifting figure across Europe? 
\end{enumerate}
\fi


\begin{figure}[h!]
\centering
\noindent \includegraphics[width=1\textwidth]{figures/scan7May2023vitis.png}
\caption{Drawing of potential conceptual figure highlighting \emph{Vitis} diversity at species level (that's supposed to be a Northern hemisphere map showing species on each continent), variety level, then clonal level, and maybe also try to show rootstock x scion, but not sure.}
\label{fig:vitislevels}
\end{figure}

\newpage
\begin{figure}[h!]
\centering
\noindent \includegraphics[width=1\textwidth]{figures/scan21Apr2023_historyfig.png}
\caption{Drawing of potential conceptual figure highlighting times of changes in winegrowing diversity. Could definitely use help with best refs to gather dates (with uncertainty) and suggested improvements.}
\label{fig:history}
\end{figure}


\newpage
\begin{figure}[h!]
\centering
\noindent \includegraphics[width=1\textwidth]{figures/rootstockxscionexpts.pdf}
\caption{Map of locations of rootstock x scion experiments conducted to date based on an ISI search of ALL FIELDS = [rootstock* AND scion* AND grape* ]. This returned 530 refs. If we like the idea of this figure I can subsample 100-200 of them and map locations (here I mapped just the first 15 of that subsample).}
\label{fig:rootstockxscionexpts}
\end{figure}

\newpage
% This was BELOW \end{document} in vitisreview_ncc_wbbl.tex
\begin{figure}[h!]
\centering
\noindent \includegraphics[width=0.6\textwidth]{..//grapesreview/figures/vardivblocks/diversityblocks_take2.png}
\caption{Conceptual diagram of how variety diversity blocks---if planted---could provide sampling of many varieties across different climates. Here we show conceptually how scale matters to the utility of such diversity blocks: Each circle represents a diversity block of 20 varieties selected by an individual  vineyard (see main text for further discussion). Because varieties would likely overlap across vineyards with overlap generally increasing for climatically (often also geographically) closer vineyards---such diversity blocks would provide data for each variety across several climates. Further, if vineyards follow our suggestion to include a diverse selection of varieties then---across all blocks---hundreds of varieties in total would be planted across a large region, such as that depicted here. Together, data pooled across diversity blocks would be extremely powerful for projecting which varieties will grow well in different regions as climates shift in the future.}
\label{fig:divblocks}
\end{figure}

\newpage
\section{Not figures}
But I thought this was interesting; same data we could use for Fig. \ref{fig:rootstockxscionexpts}.
\begin{figure}[h!]
\centering
\noindent \includegraphics[width=1\textwidth]{figures/rootstockpubyears.pdf}
\caption{Count of rootstock x scion experiments conducted to date based on an ISI search of ALL FIELDS = [rootstock* AND scion* AND grape* ], grouped by year after removing a few records from grapefruit and citrus studies. }
\label{fig:rootstockxscionexptsyears}
\end{figure}

\end{document}

An extensive study of the genetic diversity within seven French wine grape variety collections
Highlights that we don’t have phenotypic info!


\section{Outline}
\begin{enumerate}
\item Short intro on climate change and genetic diversity
\item Review what diversity is (levels/types etc.); aim to organize this around history ...
\begin{enumerate}
\item New genomic evidence suggests two centers of domestication in Europe
\item Until the 1500s we had just vinifera crossings for the most part and no real focus on anything else [Varieties]
\item Then big focus on North American \emph{Vitis}, which caused Phylloxera and an even bigger focus on North American \emph{Vitis} [Species ... starting with rootstock focus]
\item Clonal selection in the 70s (which was also likely cleaning out of disease!) [Clones]
\item Bad hybrid period to better times today? [Hybrids/species] 
\item ADD in: Subspecies (No one cares about this maybe or no diversity is left?)
\end{enumerate}
\item Historical review of what diversity have we lost (and when) along this path and why it matters ... 
\begin{enumerate}
\item History of loss
\begin{enumerate}
\item Prehistory rise of  \emph{Vitis vinifera} and loss of other European  \emph{Vitis} diversity?
\item Varieties with phylloxera ... some remaining variety diversity -- may be locked in Chile and other places that were hit by Phylloxera at different times or never were... 
\item US (and maybe Asian species) 
\end{enumerate}
\item Where is diversity today? (We're continuing to lose diversity.)
\item What do these losses mean for our ability to grow great grapes in the future?
\end{enumerate} 
\item Why diversity matters: Because of the the traits we care about ... Link this text to Figure: organize the levels of diversity and highlight where we have a lot of important variation -- for example, at variety level we do have phenological diversity and maybe some heat and drought tolerance, but we have little pest/disease tolerance ... organize into two suites of traits:
\begin{enumerate}
\item What are the traits we care about in general? 
\begin{enumerate}
\item Berry color and other berry stuff 
\item Phenology
\item Crop load and crop consistency
\item Disease tolerance/resistance
\item Pest tolerance/resistance (especially with rise of organic farming)
\item Cold tolerance
\end{enumerate}
\item What are the traits we care about especially with climate change?
\begin{enumerate}
\item Phenology
\item Drought tolerance
\item Heat tolerance
\item Pest/disease tolerance again
\end{enumerate}
\end{enumerate}
\item ... but, environmental variation matters a lot: Interaction between genetic diversity and climate determines a number of traits (phenology as example)
\begin{enumerate}
\item Var x Climate interactions
\item Rootstock x Scion interactions
\item R x S x C interactions  (and need for global data sharing)
% \item Plasticity in traits (never really sure what that meant with phenology, but maybe the idea that the forcing cue could vary based on environmental conditions. 
\end{enumerate} 
\item Future is ... combining these levels of diversity in models so we can understand response surfaces (since the environment matters so much) and start predicting them without needing data on every different grapevine. ... 
\item Major questions that we need to answer, and which flow from above
\begin{enumerate}
\item What is the scale of variation in major traits at each level (clonal/variety/species)?
\item What is the scale of variation due to above interactions? And .. are international varieties unique in strong variety responses and weak V x C so get similar responses across terroir? 
\item What tools can help with forecasting at the genetic level (i.e., unmeasured varieties or such)? 
\begin{enumerate}
\item Trait correlations: What suites of traits are correlated? Phenological phases, drought x heat tolerance, etc.?
\item Gene x trait models: with more genetic data and models that deal better with reticulate phylogenies we can mapping traits on them and test for this
\item We also need better models of what has driven \emph{Vitis} speciation (e need a lot more info on native \emph{Vitis} to breed in useful traits)
\end{enumerate}
\end{enumerate}
\item How can we get better data/answer these questions?
\begin{enumerate}
\item Need to embrace diversity across these levels, which is what research collections do % Focus on lost diversity could give the paper an angle -- suggest we stop fighting over what diversity is most important and focus on it more equally... % okay, so maybe that is pretty hooky, but I am still throwing it out there. 
\item More work on partitioning variation across levels of diversity 
\item Better sharing across research groups and countries of R x S and similar trials, maybe end this on need fund that is specific to synthetic work? Which leads into another new iniative ... 
\item Grower initiatives
\end{enumerate}
\end{enumerate}

\section{Other stuff}
Dangers of limited genetic diversity in grapevines (ASEV talk)
\begin{enumerate}
\item What I covered in talk .... 
\begin{enumerate}
\item Review levels of diversity (species, varieties, clones ... hybrids)
\item Why we need/how we use diversity (rootstock, disease, cold, flavor, heat tolerance etc.)
\item Where is diversity now?
\item How diversity matters to pruning/disease etc.
\item Phenology x climate hazards
\item Variety diversity matters to projections
\item Research collections across the world and investing more in them (connect more to disease/pest)
\item Grower initiatives 
\end{enumerate}
\item What I did not cover but could ...
\begin{enumerate}
\item Climate change will not go away
\item GDD across regions?
\item Vassal and how narrow diversity we mainly use is
\end{enumerate}
\end{enumerate}



