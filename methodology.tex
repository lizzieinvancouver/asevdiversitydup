\documentclass[11pt, oneside]{article}   	% use "amsart" instead of "article" for AMSLaTeX format
\usepackage{geometry}                		% See geometry.pdf to learn the layout options. There are lots.
\geometry{letterpaper}                   		% ... or a4paper or a5paper or ... 
%\geometry{landscape}                		% Activate for rotated page geometry
%\usepackage[parfill]{parskip}    		% Activate to begin paragraphs with an empty line rather than an indent
\usepackage{graphicx}				% Use pdf, png, jpg, or eps§ with pdflatex; use eps in DVI mode
								% TeX will automatically convert eps --> pdf in pdflatex	
\usepackage{hyperref}				% you can use \url{} with this package
	
\usepackage{amssymb}

%SetFonts

%SetFonts


\title{Methodology Asevdiversity}
\author{Lizzie Wolkovich and Christophe Rouleau-Desrochers}
\date{\today}

\begin{document}
\maketitle
\section{Steps}
\subsection{Step 1 – Subset of papers}
We conducted a literature review of papers that explored experiments on rootstock x scion interactions in grapevines. Using an ISI Web of Science (Core Collection) search with the query:

rootstock* AND scion* AND grape* (ALL FIELDS)

we initially retrieved 530 references. From those, we randomly selected 140 articles to review for relevant data. We excluded articles focused on species other than grapes (e.g., grapefruit, kiwis), and a limited set we could not obtain from our university library. We scraped data from the remaining articles.
% 1Nov2023: Below feels like a short form of the next section, so I would cut it but replace with topic sentence (to optimize short x clarity). 
% 4Nov2023:Perfect. Indeed it's redundant.
% We scrutinized these articles to determine their relevance to our research, noting their provenance and the number of rootstocks and scions employed in each experiment. Some of the selected articles.

\subsection{Step 2 – Data scraping}

% 1Nov2023: I'd give the website we used to convert degrees \url{website}
% 4Nov2023URL is ok I believe
We recorded the following from each article (when possible): location, number of rootstocks and number of scions. We recorded the geographical locations in decimals where each study was conducted. If the location was given in hour/degree/minutes, we converted it in decimals to ensure consistency (using \url{https://www.fcc.gov/media/radio/dms-decimal} ). In cases where explicit coordinates were absent, we derived decimal coordinates based on the most precise location information available as follows: (1) if the study mentioned the name of the vineyard or campus where the experiment occurred, we used those location details, or (2) when no location was given, we estimated the location based on first author’s affiliation. We report the method that we used for location in the \emph{how} column of rootstockscionlatlon dataset. % 1Nov2023: spreadsheet is sort of an excel term, and you'll likely upload as csv so I would avoid the term

Additionally, some of the papers retrieved from the ISI search were review articles. For these, we extracted relevant data if it was explicitly presented within the article. In instances where certain information was missing, we conducted further searches to locate and include the specific studies mentioned in the review paper. We documented the ID number of the review paper that led us to the referenced article in the \emph{source\_paper\_id} column within the rootstockscionlatlon dataset.

% 1Nov2023: Subsequently -> next (when you can, use simpler terms usually)
% 1Nov2023: This is part of scraping so I moved it up. Also, to be consistent I would give the column names here also. 
% 4Nov2023: Perfect. Column names are added here. 
Next, we determined the number of rootstocks and scions utilized in each experiment. When the data was available, we noted how many of each were used in the \emph{n\_rootstock} and \emph{n\_scions} columns.

\subsection{Step 3 – Mapping rootstocks / scion data}
Then using Plotly package in R, we created a choropleth map where the color gradient corresponds to the number of articles investigating rootstock x scion interactions in each country. We also adjusted the size gradient of the data points to reflect the number of rootstock varieties used in each study. The interactive version (HTML) was all created in R and the static version (SVG) was partially created in R. A second legend displaying the meaning of the dot sizes was created in Adobe Illustrator.

\section{Data Table, Image, and Other Data Details}
\subsection{Other entity 2}
\textbf{choropleth\_script.R} % 1Nov2023: nice to post code! Actually -- the whole set of files and description here is very good. And way to go on latex! I am impressed how much latex code is here and all functional. Well done. 

Description:
Code, to create a choropleth map of rootstock x scion studies across the world. =
\subsection{Other entity 2}
\textbf{isi5May2023.bib}

Description:
Isi return of all 530 references
\subsection{Other entity 3}
\textbf{rootstockscionlatlon.csv}

Description:
Rootstock x scion interaction data set in which locations and rootstock and scion numbers were extracted.



\end{document}

